\chapterimage{data}
\chapter{Data}\label{sec:data}

This unit introduces the notion of digital \concept{data} and discusses multiple non-trivial aspects
that need to be considered to read/receive, manipulate and write/send them.

\section{Digital media}\label{sec:data:media}

Digital images typically consist of one or more matrices of \concept{integer} data~\cite{chua_data_representation}.
%
Each of these matrices is a \concept{band} (also known as \concept{component}).
Each element (cell) of the matrix is a \concept{pixel}. Indices are often referred to
using their spatial position $(x,y)$ and the index of their band, \eg, $I_{x,y,z}$.
If only one band is used, pixels can be referred to using only the spatial indices, \eg, $I_{x,y}$.

One-band images are \concept{grayscale} (sometimes loosely referred to as \concept{monochrome}).
In them, pixels with low values are often dark (black being the lowest possible value).
Conversely, pixels with high values are bright (white being the highest possible value).

When more than one band is present, each band typically contains information about a
limited range of light \concept{wavelengths}. For instance, $3$-band images are often \concept{RGB},
meaning that bands describe the \concept{intensity} of red, green and blue light, respectively.
This means that the maximum pixel value in these components represent the maximum intensity
of red, green and blue light, respectively.

There also exist images with more than $3$ bands.
Tens, hundreds and even thousands of bands can be present, \eg,
when produced by \concept{multispectral} and \concept{hyperspectral} sensors onboard satellites,
or by special types of microscope. In this case,
each band represents a narrow band of wavelengths, and the maximum pixel value
indicates the maximum intensity of photons of those wavelengths.

Many other types of digital media exist, including \concept{video}, \concept{audio}, electrocardiograms (ECGs),
seismographic and other time series, \etc. It is often useful to consider these data as \concept{samples} of
1D functions (\eg, audio channels, ECGs), 2D functions (\eg, monochrome images),
3D functions (\eg, color images and monochrome video), 4D (\eg, color video)
and even higher-dimensionality data (\eg, multispectral video and database records).
In almost all cases, these digital data are also integer values
although they can also be \concept{floating point}/real.

\section{Data representation, raw formats and metadata}

When storing data in a file or transmitting them over the network,
samples need to be arranged in a 1D pattern, \ie, they need to be \concept{serialized}.
\conceptRef{raster}{Raster},
band-sequential \concept{BSQ}, band-interleaved by line \concept{BIL} and by pixel \concept{BIP}
data orders~\cite{collins_understanding_rasters,loc_bsq,loc_bil,loc_bip} are the most prevalent.

Before applying compression, samples are often (temporarily) stored using \concept{fixed-length} representations,
hereafter referred to as \concept{raw} representations or raw data.
%
Longer (higher \concept{bitdepth}) representations permit
larger \conceptRef{dynamic range}{dynamic ranges}, \ie, a larger range of possible
values for the samples.
%
For instance, an image in which pixels can take one of $256$ possible values may
be represented using $8$-bit samples. Another image in which pixels can take $1024$ possible values
(thus allowing many more colors or shades of gray and a better quality)
would require at least $10$ bits per sample when stored in raw format.

Typically, samples are stored using full \conceptRef{byte}{bytes} ($8$, $16$, $32$ and $64$ bits)
for efficiency reasons,
so that programs can read them directly without having to process individual \conceptRef{bit}{bits},
\eg, from the most significant bit (\concept{MSB}) to the least significant one (\concept{LSB}).
%
Regardless of their bitdepth, samples can be \concept{signed} (\eg, for audio) or \concept{unsigned}
(\eg, most image data).
%
For multi-byte samples, it is necessary to specify the byte ordering:
\concept{big endian} (\aka network order), \concept{little endian} (used by many CPU architectures).

Raw formats are not compressed and require custom software tools (\eg, Imagej~\cite{rasband_imagej}).
Thus, they are intermediate formats that simplify the input/output of other tools, particularly
for data \concept{compression}. Since raw formats contain no \conceptRef{header}{headers},
\concept{metainformation} about the sample \concept{geometry} and \concept{data type} must be stored separately as
\concept{side information}, \eg, as a part of the file name or in a separate file.


\section*{Further reading and practice}
\vspace{0.25cm}
\begin{remark}
    Check the ``References'' section of the Campus Virtual if you have trouble finding any reference.
\end{remark}

\begin{itemize}
    \item \cite{chua_data_representation}: how sample values are stored in disk/memory.
    \item \cite{siegfried_bytes}: how digital data is structured in bits and bytes, and their meaning.
    \item \cite{collins_understanding_rasters}: basics of digital images ("rasters") and their properties.
        See in particular the "General properties of rasters" and "Raster Graphics" sections.
    \item \cite{loc_bsq,loc_bil,loc_bip}: further info on BSQ, BIL and BIP data orders.
\end{itemize}

\begin{exercise}
Complete the ``Quiz: Data I/O'' activity on the Campus Virtual.
\end{exercise}


\begin{exercise}
Grab the mandrill image sample (\url{mandrill-u8be-3x512x512.raw}).
\begin{itemize}
    \item Open it, \eg, with ImageJ, and visualize its colors.
    \item Read the red, green and blue pixel values at position x=100, y=200 of the mandrill image
    and print the values.
    \item Count the number of bits set to 1 in the image. Print that number.

    \item Save a version of the mandrill image after setting all the most significant bits (MSBs)
          of the green component to 1 (name it \url{mandrill_t93_u8be-3x512x512.raw}).
    \item Save a version of the mandrill image after setting all the least significant bits (LSBs)
          of the blue component to 0 (name it \url{mandrill_t94_u8be-3x512x512.raw}).
    \item Save a version of the original mandrill image using signed integers,
          16 bits per sample. Also, set the pixel at position $(0,0,0)$ to \texttt{0x0000} and
          the pixel at $(1,0,0)$ to \texttt{0xFFFF}. Save it as \url{mandrill_t95_s16be-3x512x512.raw}
          and open it with ImageJ.
\end{itemize}
\end{exercise}


% ## Unit practice
%
% The following activities are available on the Campus Virtual to practice the concepts of this unit:
%
% * Unit C1 quiz: questions about raw file size, raster ordering and endianness. You should be able to obtain the maximum score (you can try as often as necessary).
% * Unit C1 task list: submit Task 9 described in this list to practice data I/O programming and receive feedback.
%
