\chapterimage{doors}
\chapter{Entropy coding}\label{sec:coding}

The main goals of this chapter is to formalize the most important concepts about
codes, to link the notions of entropy and compressibility, and understand the
role of entropy coding in data compression pipelines.

\section{Codes and compressibility}
Source coding is the process of transforming a \concept{message}, \ie,
a sequence of input symbols $s_1,\,\ldots,\,s_N$ into
a \concept{compact}, generally binary, \concept{representation}. To be useful,
this process must be \concept{invertible}, \ie, it must be possible to
(uniquely) retrieve the original symbols from the compact representation.

\conceptRef{variable-length code}{Variable-length codes} transform each
input symbol $\sigma_i$ into a \concept{codeword} $C(\sigma_i)$,
whose \conceptRef{codeword length}{length} $L_i$ depends on that symbol's
probability.
The code can be expressed as a function, \eg, \mbox{$C: \alphabet \longrightarrow \{0,1\}^{*}$}.

To reduce the compressed data size, the most probable symbols are
assigned codewords of smaller lengths, and
vice-versa~\cite[\S 2.4, \S 2.4.1, \S 2.4.2]{sayood_introduction}.
%
Given the length $L_i$ of each symbol $\sigma_i$,
the code's \concept{expected length} of a code is obtained as
$L_C = \sum_i \Pri L_i$~\cite[\S 5.1]{cover_elements}.

\conceptRef{prefix code}{Prefix codes} are those for which no $C(\sigma_i)$
is the prefix of another $C(\sigma_j)$.
This property allows both \conceptRef{instantaneous decoding}{instantaneous}
and \concept{unique decoding}, making it highly desirable for practical data compression.

Shannon proved that a code $C$'s expected length $L_C$ must be
at least $\entropy(\source)$~\cite[\S 5.4]{cover_elements}. One way to reach this
result is by using Kraft-McMillan's inequality ($\sum_i D^{-L_i} \le 1$~\cite[\S 2.4.3]{sayood_introduction}) and Lagrangian optimization:
\begin{eqnarray*}
J &=& \sum_i \Pri L_i + \lambda \sum_i D^{-L_i}, \\
0 = \frac{\partial J}{\partial L_i} &=& \Pri + \lambda D^{-L_i} \cdot (-1) \cdot \log D
\Rightarrow D^{-L_i} = \frac{\Pri}{\lambda \log D}, \\
1 \ge \sum_i D^{-L_i} &=& \frac{1}{\lambda \log D} \sum_i \Pri = \frac{1}{\lambda \log D}
\Rightarrow \lambda \ge \frac{1}{\log D}, \\
D^{-L_i} &=& \frac{\Pri}{\lambda \log D} \le \Pri \Rightarrow L_i \ge - \log \Pri
\Rightarrow L_C = \sum \Pri L_i \ge \entropy(\source).
\end{eqnarray*}

It is always possible to find a prefix code that satisfies
$L_i = \left\lceil -\log_2(\Pri)\right\rceil$~\cite[\S 2.4.2, \S 2.4.3]{sayood_introduction},
\eg, using \concept{Huffman's algorithm} (see below).
Thus, given a source with entropy $\entropy(\source)$, it is always possible to
find a prefix code that satisfies $\entropy(\source) \le L_C \le \entropy(\source) + 1$.
%
Furthermore, by coding
$M$ symbols at once, the bound can be tightened as much as desired:
\mbox{$\entropy(\source) \le L_C \le \entropy(\source) + 1/M$}
(and $\displaystyle \lim_{M\rightarrow\infty} L_C = \entropy(\source)$).
\conceptRef{arithmetic coding}{Arithmetic coders} (also see below) are a prime example of this.

A poorly designed code will have a larger expected length than
the data's actual entropy. In particular, this will happen if we optimize a code
for a probability distribution $P$, but the source $\source$ has a different
distribution $Q$. The \concept{Kullback-Leibler} (KL) ``distance'',
\aka\ \concept{relative entropy} or \concept{cross entropy}, provides
the expected \concept{overhead}, \ie, the extra bits needed per symbol on
average~\cite[\S 2.3, Theorem 5.4.3]{cover_elements}. This distance is defined
as $$D(P \parallel Q) = \sum_i \Pri \log_2(\Pri/Q(\sigma_i)),$$
using the convention $0 \log_2(0/0) = 0$, $0 \log_2(0/q) = 0\ \forall q$, and
$p \log_2(p/0) = \infty\ \forall p > 0$.

\section{Practical trade-offs}~\label{sec:coding:trade_offs}

Entropy coding is a key stage of most compression \conceptRef{pipeline}{pipelines}.
This stage is \conceptRef{lossless compression}{lossless},
although it can be preceded by other \conceptRef{lossy compression}{lossy}
stages to enhance compression at the cost of introducing some distortion in the data
(see Chapters~\ref{sec:quantization} and~\ref{sec:rate_distortion}).

In order for those pipelines to be practical, the algorithm selected for entropy coding
must strike an acceptable trade-off between \concept{efficiency} (defined as $\eta = \frac{\entropy(\source)}{\textrm{compressed bitrate}}$) and \concept{computational complexity}.

On one end of the spectrum, there are relatively simple algorithms designed to run fast
requiring few resources (including energy).
%
Often, these algorithms code each input sample
independently, and work best when these samples have been previously decorrelated
(\eg, as described in Chapter~\ref{sec:prediction}).
%
Their trade-off is that they attain $\eta=1$ only for limited classes of probability distributions.
%
Prime examples of this approach are:
\begin{itemize}
    \item \concept{Huffman's algorithm}~\cite[\S 3.2--3.3]{sayood_introduction},~\cite[\S 2.2.1]{taubman2002jpeg2000},~\cite[\S 5.6--5.7]{cover_elements}.
    \item \concept{Golomb-Rice} codes~\cite[\S 3.5--3.6]{sayood_introduction}, employed for instance in the CCSDS~121.0-B-3 standard~\cite[\S 3.2--3.3]{ccsds121x0b3}.
    \item \concept{Run-Length Encoding} (RLE)~\cite{golomb_rle}.
\end{itemize}

On the other side of the spectrum, there are algorithms designed to consistently achieve high efficiencies approaching $\eta=1$
for most distributions, generally at the cost of higher computational complexities. To do so, instead of coding individual symbols,
they process a sequence of symbols and produce a single output codeword. Traditionally,
the main paradigm used for this purpose is
\conceptRef{arithmetic coding}{Arithmetic Coding}~\cite[\S 4]{sayood_introduction},\cite[\S 2.3]{taubman2002jpeg2000},
sometimes referred to as \concept{range coding}. Some prime examples of this paradigm are:
    \begin{itemize}
    \item the \concept{CABAC} entropy coders of the popular video codecs H.264 (AVC, 2003), H.265 (HEVC, 2013) and H.266 (VVC, 2017),
    \item the \concept{MQ} entropy coder of JPEG~2000,
    \item the \concept{PAQ} family of entropy coders, among others.
    \end{itemize}

More recently, thanks to Jarosław Duda's contributions~\cite{duda_asn}, entropy coding
has been shifting towards \concept{Asymmetric Numeral Systems} (ASNs), specially its
\concept{Finite State Entropy} (FSE) realizations. Some important examples of ASN are:
    \begin{itemize}
    \item \concept{Zstandard} (\aka\ zstd)~\cite{collet_zstd},
    \item LZFSE~\cite{lzfse},
    \item \concept{JPEG~XL}~\cite{jpeg_xl}, and
    \item CCSDS~123.0-B-2~\cite{ccsds123x0b2}, among many others.
    \end{itemize}

\section{Universality and Adaptativity}\label{sec:coding:adaptativity}

The most basic versions of the entropy coders mentioned in Chapter~\ref{sec:coding:trade_offs}
rely on two main assumptions: the probability distribution of the source \source\ is known \textit{a priori},
and that probability does not change throughout the compression process. Neither of these assumptions hold
generally in practice, and a \concept{two-pass} approach is not necessarily practical.

In the late 70's, Ziv and Lempel introduced the \concept{LZ77}~\cite[\S 5.4.1]{sayood_introduction}~\cite{lz77}
and \concept{LZ78}~\cite[\S 5.4.1]{sayood_introduction} algorithms,
often referred to as \concept{zip} compression, which dynamically build a dictionary of previously
seen sequences. As new symbols are input, symbols can be efficiently represented referencing that dictionary.
These algorithms seeded a vast class of compressors that is still relevant and growing today.

Algorithms such as Huffman and Arithmetic Coding can be extended so that they use \concept{adaptive} probability
models. While still reading each input sample only once, these algorithms change their estimation of each symbol's
probability based on previously observed samples:
\begin{itemize}
\item  With Huffman, for instance, input samples can be divided into blocks, and
the statistics gathered for one block can be used to generate the code table for the next block.

\item With an Arithmetic Coder, each sample can be used to update the probability range division
characteristic of this algorithm. It is also typical to employ contexts to further refine the
probability model and better exploit any redundancy present in the inputs,
like CABAC does, and as previously mentioned in Chapter~\ref{sec:info:surprise}.
\end{itemize}

\section*{Further reading and practice}
\vspace{0.25cm}

\begin{itemize}
\item Sayood's book~\cite{sayood_introduction} provides invaluable and very accessible insight on most topics of this unit,
particularly in Sections 2.4, 3.2, 3.5 and 3.6.

\item Taubman and Marcellin's book~\cite{taubman2002jpeg2000} offers a somewhat more rigorous (not that Sayood's isn't!) treatment of these algorithms, often from the perspective of image compression, in Sections 2.2, 2.3, and 2.4.

\item A thorough analysis of universal coders (including Arithmetic and LZ coding) can be found in~\cite[\S 13]{cover_elements}.

\item The book by McAnlis and Haecky provides a very clear description of tANS, one of the main
variants of asymmetric numeral systems~\cite[\S 5.5]{mcanlis_understanding}. Further mathematical
insight, including its relation to computer security (cryptography), can be found in Duda's paper~\cite{duda_asn}.

\end{itemize}


\begin{exercise}
Consider the four following codes for a four-symbol alphabet $\alphabet = \{a,b,c,d\}$:

\begin{center}
\begin{tabular}{ccccc}
\toprule
\textbf{Code $\mathbf{C}$} & $\mathbf{C(a)}$ & $\mathbf{C(b)}$ & $\mathbf{C(c)}$ & $\mathbf{C(d)}$ \\
\toprule
$C_1$ & \texttt{0} & \texttt{0} & \texttt{0} & \texttt{0} \\
$C_2$ & \texttt{0} & \texttt{010} & \texttt{01} & \texttt{10} \\
$C_3$ & \texttt{10} & \texttt{00} & \texttt{11} & \texttt{110} \\
$C_4$ & \texttt{0} & \texttt{10} & \texttt{110} & \texttt{111} \\
\bottomrule
\end{tabular}
\end{center}

\begin{itemize}
\item Which of the codes are usable?
\item Which compresses best?
\end{itemize}

\end{exercise}


\begin{exercise}
What's the minimum and maximum overhead of Huffman coding?
When's the minimum overhead achieved? How about the maximum?
\end{exercise}

\begin{exercise}
Is RLE better suited for high-entropy or low entropy sources?
\end{exercise}

\begin{exercise}
What type of distribution are Golomb codes optimal for?
\end{exercise}

\begin{exercise}
Find the distribution of English characters. Based on that probability model,
encode your name using
\begin{enumerate}
    \item Huffman coding
    \item Arithmetic coding
\end{enumerate}
If needed, romanize your name so that it contains only English caracters (\eg, using Pinyin).
\end{exercise}

\begin{exercise}
Find implementations of a couple of entropy coders among the described in this section, \eg, at:
\begin{itemize}
\item \url{https://github.com/nayuki/Reference-arithmetic-coding}
\item \url{https://github.com/nayuki/Reference-Huffman-coding}
\item \url{https://github.com/Cyan4973/FiniteStateEntropy}
\item \url{https://github.com/miguelinux314/pyac}
\end{itemize}

Apply them to the \url{mandrill-u8be-3x512x512.raw} image and compare the actual compressed
size against its entropy.
\end{exercise}

