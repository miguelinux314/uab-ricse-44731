% Information about the course and the guide

\chapterimage{classroom_background.jpg}

\chapter{Intro: Data Transmission and Security}\label{sec:course}

\section*{The subject}

The goal of this subject is to help you acquire engineering and research competences
in the area of data transmission.
%
The course focuses on the efficiency and security
of this process, with special attention to data compression.

Successful students will be able to design, analyze and evaluate data transmission solutions
in realistic scenarios. To do so, they will be expected to understand the theoretical
principles behind those systems (particularly surrounding the area of Information Theory),
and to develop their own software toolboxes to perform quantitative analysis and evaluation.


\section*{The guide}
This guide is designed to help students find and organize new information related to the course.
This information should be actively and individually gathered by each student.
%
The guide is \textbf{not} a book, a reference material or even a complete set of class notes, and
\textbf{it is not intended to substitute your own notes}. You are expected to complete them with
the proposed materials, including but not limited to~\cite{sayood_introduction,taubman2002jpeg2000,mcanlis_understanding}.

\begin{remark}
    Full bibliographic entries are provided at the end of the document.
    Most of these are all freely accessible using your UAB credentials (see the Campus Virtual).
\end{remark}

\section*{Sessions}

The course is structured in $7$ incremental units. Each one contains one or more of the following session types:

\begin{itemize}
\item \textbf{Discover} sessions: students will be exposed to new concepts via oral discussions and selected materials.
Short, individual exercises will be proposed as homework until the next session.

\item \textbf{Deepen} sessions: after a Discover session, students will discuss their solution to the proposed exercises
and explore further, progressively more complex scenarios in one or more Deepen sessions.

\item \textbf{Develop} sessions: at the end of each Discover-Deepen-Develop unit, one session will be devoted to autonomous,
semi-supervised practice of the concepts pertaining to that (or previous) units.
\end{itemize}


\begin{center}
\begin{minipage}{0.75\linewidth}
\begin{itemize}
 \item[8$\times$] 2h {\color{color1}\textbf{Discover sessions}}
 \item[9$\times$] 2h {\color{color2}\textbf{Deepen sessions}}
 \item[7$\times$] 2h {\color{color3}\textbf{Develop (challenge) sessions}}
 \item[Unlimited] \textbf{\color{darkgray}Office hour sessions} (\textit{tutories}), individual or group: send email
 \item[92h] {\textbf{\color{gray}Autonomous work} (total, expected)}
\end{itemize}
\end{minipage}
\end{center}
A temporal plan describing the units and the session types is available in the Campus Virtual of the subject.

\section*{Evaluation}

The final course grade is given by the arithmetic mean of the $7$ tests taken during these many \textit{Develop} sessions, \ie,
%
$ \frac{1}{7} \sum_{i=1}^7 D_i$,
%
where $D_i$ is the $i$-th test score.
